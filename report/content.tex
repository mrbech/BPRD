\section*{Introduction}
The following report is Martin Rønning Bech's answer to the Programs as Data
(BPRD) E2013 examination. It is split into 4 sections for each of the
assignments, with each question having a subsection. So the solution for
assignment 3 question 2 can be found in section 3.2. The code snippets for each
of the questions can be found in the corresponding question section, the code
have been cut out of the file and \texttt{...} is used to denote omitted code
sections. Two files can found in the Appendix that I use for compiling the
MicroC compiler, the first is a main function and the second is my make file,
for a more detailed description of each file see the appendix.

The code have been compiled on Linux Ubuntu 12.10 using F\# version 3.0 (Open
Source Edition), FSharpPowerPack version 4.0, javac version 1.7.0\_45 and run
using Java 1.7.0\_45 from Oracle.

\tableofcontents
\pagebreak


\section{Opgave 1}
\subsection{Implementation}
\subsubsection*{CLex.fsl}
The changes needed for \texttt{CLex.fsl} consists of adding the "for", "to" and "do"
keywords to the lexer.
\begin{fs}
let keyword s =
    match s with
    | "for"     -> FOR
    | "to"      -> TO
    | "do"      -> DO
    ...
\end{fs}
\subsubsection*{CPar.fsy}
The \texttt{CPar.fsy} parser then needs to be extended with the \texttt{FOR},
\texttt{TO} and \texttt{DO} tokens. It is then possible to add the new for-to
loop statement in the \texttt{StmtM} and \texttt{StmtU} declaration. I have
chosen to directly parse the for-to loop into already existing abstract syntax,
therefor I have made no changes to \texttt{Comp.fs} or \texttt{Absyn.fs}.

The for-to loop is translated as shown in the assignment by the parser: 
\begin{enumerate}
    \item Inside of the block it first declares the counting variable and
        assigns it to the initial value given by the user (\texttt{Block},
        \texttt{Dec} and \texttt{Assign}).
    \item It then inserts a while loop that loops until the counting variable
        have reached the limit that the user have set (\texttt{While},
        \texttt{Prim2("<=",...)} and \texttt{AccVar}).
    \item Inside the while loop we insert the users statement and end the while
        loop by increment the counting variable (\texttt{Stmt}, \texttt{Assign},
        \texttt{Prim2("+",...)}, \texttt{AccVar} and \texttt{CstI(1)}).
\end{enumerate}
The code snippets for \texttt{CPar.fsy} can be found bellow:
\begin{fs}
%token FOR TO DO
...
StmtM:
...
  | FOR NAME ASSIGN Expr TO Expr DO StmtM { Block[Dec(TypI, $2); Stmt(Expr(Assign(AccVar($2), $4))); Stmt(While(Prim2("<=", Access(AccVar($2)), $6), Block[Stmt($8);Stmt(Expr(Assign(AccVar($2), Prim2("+", Access(AccVar($2)), CstI(1)))))]))] }
...
StmtU:
  | FOR NAME ASSIGN Expr TO Expr DO StmtU { Block[Dec(TypI, $2); Stmt(Expr(Assign(AccVar($2), $4))); Stmt(While(Prim2("<=", Access(AccVar($2)), $6), Block[Stmt($8);Stmt(Expr(Assign(AccVar($2), Prim2("+", Access(AccVar($2)), CstI(1)))))]))] }
\end{fs}
\subsection{Test}
Below is my test example code. It uses the for-to loop to print the numbers 0 - 10
and ends by printing a newline. Resulting in an output like the following:
\texttt{0 1 2 3 4 5 6 7 8 9 10}
\begin{ccode}
void main() {
  for i=0 to 10 do 
    print i;

  println;
}
\end{ccode}
\subsection{Result}
To run the example I use two extra file, one called mcc.fs and a make file both can
be found in the appendix with a short description of what they do. With the two
files in the same directory as the MicroC code and the example code stored in
\texttt{examples} directory I can do the following to test my example:
\begin{bashcode}
% make
% ./mcc.exe examples/ex1.c examples/ex1.out
% javac Machine.java
% java Machine examples/ex1.out
\end{bashcode}
This gives me the expect output of:
\begin{bashcode}
% java Machine examples/ex1.out
0 1 2 3 4 5 6 7 8 9 10

\end{bashcode}

\section{Opgave 2}
\subsection{Implementation}
\subsubsection*{CLex.fsl}
The changes needed for \texttt{CLex.fsl} consists of adding the plus equal and
minus equal tokens to the lexer.
\begin{fs}
rule Token = parse
...
  | "+="            { PEQ }
  | "-="            { MEQ }
...
\end{fs}
\subsubsection*{CPar.fsy}
The \texttt{CPar.fsy} parser will also need the two new tokens. We also need to
add precedence for the two new tokens, they are non-associative similar to
greater-than, less-than etc. The expressions can then be added to the
\texttt{ExprNotAccess} declaration as seen bellow, being parsed into two new
abstract syntax expressions.
\begin{fs}
%token PEQ MEQ
...
%nonassoc GT LT GE LE
%nonassoc PEQ MEQ 
%left PLUS MINUS
...
ExprNotAccess:
...
  | Access PEQ Expr                     { PlusEq($1, $3)      } 
  | Access MEQ Expr                     { MinusEq($1, $3)     } 

\end{fs}
\subsubsection*{Absyn.fs}
In the abstract syntax found in the \texttt{Absyn.fs} file we add the two new
expression to the \texttt{expr} declaration. They both consists of a tuple of
type access and expression.
\begin{fs}
and expr =                                                         
  | PlusEq of access * expr          (* x+=e or *p+=e or a[e]+=e    *)
  | MinusEq of access * expr         (* x-=e or *p-=e or a[e]-=e    *)
\end{fs}

\subsubsection*{Comp.fs}
In the compiler implementation found in \texttt{Comp.fs} we add two new match
cases in \texttt{cExpre}. The two expressions have a similar implementation, and
compiles to the following:
\begin{enumerate}
    \item Puts the left hand side's stack index on the stack (\texttt{cAccess})
    \item Duplicates the access on the stack, to avoid double evaluation of the
        left hand side ([DUP])
    \item Loads the left hand side value unto the stack ([LDI])
    \item Evaluates the right hand side expression (\texttt{cExpr})
    \item Adds or subtracts the expression from the accessed value ([ADD])
        ([SUB]) 
    \item Stores the result of the calculation in the left hand side's stack
        index ([STI]) (Left over from [DUP])
\end{enumerate}
The code snippet can be found bellow.
\begin{fs}
and cExpr (e : expr) (varEnv : varEnv) (funEnv : funEnv) : instr list = 
...
    | PlusEq(acc, e)  -> cAccess acc varEnv funEnv @ [DUP] @ [LDI] @ cExpr e varEnv funEnv @ [ADD] @ [STI]
    | MinusEq(acc, e)  -> cAccess acc varEnv funEnv @ [DUP] @ [LDI] @ cExpr e varEnv funEnv @ [SUB] @ [STI]
\end{fs}

\subsection{Test Example}
Below is my test example code. It starts by declaring a new array with two
elements and a variable \texttt{i}, it then initialises i to 0, the first element the
array to 10 and the second element of the array to 100.

To test that there are no double access of \texttt{i} as describe in the
exercise, we use the expression of assigning the \texttt{i} variable to it self
plus one and uses it to access and plus equal the array element with 100. We
then print the two array elements and the i, the result should be that
\texttt{10 200 1} is printed, if there is no double access and the update have
happened successfully. The minus equal is the tested in a similar fashion using
a variable \texttt{j} and should result in \texttt{10 100 1} being printed. The
test code can be seen below
\begin{ccode}
void main() {
    int arr[2];
    int i;
    i = 0;
    arr[0] = 10;
    arr[1] = 100;
    
    arr[i=i+1] += 100;
    print arr[0];
    print arr[1];
    print i;
    println;

    int j;
    j = 2;
    arr[j=j-1] -= 100;
    print arr[0];
    print arr[1];
    print j;
    println;
}
\end{ccode}
\subsection{Result}
The output is as expected \texttt{10 200 1} and \texttt{10 100 1} and a test run
can be seen below:
\begin{bashcode}
% java Machine examples/ex2.out
10 200 1 
10 100 1
\end{bashcode}

\section{Opgave 3}
\subsection{Implemetation}
\subsubsection*{Machine.fs}
The \texttt{Machine.fs} file will need to be updated with INDEX added as a new
instruction, with 26 as its code. Resulting in a need for updating the
makelabenv. The emitted instructions will also have to include a new match for
INDEX to put the CODEINDEX on the instruction list.
\begin{fs}
type instr =
    | INDEX 
...
let CODEINDEX  = 26
let makelabenv (addr, labenv) instr = 
    match instr with
    | INDEX          -> (addr+1, labenv)
...
let rec emitints getlab instr ints = 
    match instr with
    | INDEX          -> CODEINDEX  :: ints
...
\end{fs}
\subsubsection*{Machine.java}
The \texttt{Machine.java} abstract machine will also need to be updated to
support the new instruction. First it needs to be declared as the value 26 it
can then be matched in the execution loop.

The INDEX case can be found in the code below. I have implemented it very
explicitly such that it can be seen how each variable is found. First the
\texttt{a} is taken from the stack and then used to find \texttt{q}, we then
subtract \texttt{q} from \texttt{a} to calculate \texttt{n}. The array index
accessed can be found at the top of the stack, I store it in \texttt{i}. The
array index can then be compared to 0 and \texttt{n} to determine if it is out
of bounds or not. 

In the case that \texttt{i} is not out of bounds, the stack is updated to set
the second top element (\texttt{sp-1}) to \texttt{q} plus \texttt{i} the stack
pointer is then decreased by one, resulting in the stack stackindex for the
desired element being on the top of the stack. If \texttt{i} is out of bounds,
the error "Array Index Out of Bounds" will be printed to the console and the
abstract machine will exit with error code 1.
\begin{ccode}
final static int INDEX = 26;
static int execcode(int[] p, int[] s, int[] iargs, boolean trace) {
...
    switch (p[pc++]) {
      case INDEX:
        int a = s[sp-1];
        int q = s[a];
        int n = a - q;
        int i = s[sp];
        if(0 <= i && i < n){
            s[sp-1] = q+i;
            sp--;
        }else{
            System.out.println("Array Index Out of Bounds");
            System.exit(1);
        }
        break;
...
\end{ccode}
\subsubsection*{Comp.fs}
Now with the new instruction added we can change the compiler to use this new
feature. In the \texttt{Comp.fs} file in the function \texttt{cAccess} we can
change the \texttt{AccIndex} case to the following:
\begin{fs}
and cAccess access varEnv funEnv : instr list =
    match access with 
    | AccIndex(acc, idx) -> cAccess acc varEnv funEnv @ cExpr idx varEnv funEnv @ [INDEX]
...
\end{fs}

\subsection{Index Checks}
The \texttt{AccIndex} is used for any array access, where it leaves the index of
the array access on the stack top. The array access will be used for assignments
to the array such as \texttt{arr[i] = 42} where it loads the index and uses it
to save to the array. The array access will also be used in the case of just
loading the value from the array such as \texttt{x = arr[i] + 2}, where we load
the index and uses to load the value. In both cases the new index implementation
will be used like the old did, resulting in an index bound checking on both
cases.

\subsection{Test Program}
Below is my test code, the result of running it should be that \texttt{i} prints
one of the elements in the array. \texttt{i} will not go out of bounds when
\texttt{0 <= i < 3}. \texttt{j} will first store 42 in the array and then print
it, it will also not go out of bounds when \texttt{0 <= j < 3}

\begin{ccode}
void main(int i, int j) {
    int arr[3];
    arr[0] = 10;
    arr[1] = 20;
    arr[2] = 30;
    print arr[i];
    arr[j] = 42;
    print arr[j];
}
\end{ccode}
\subsection{Result}
\texttt{i=1} and \texttt{j=1} should output \texttt{20 42} because
\texttt{arr[1] = 20}. This demonstrates what happens when none of the accesses
are out of bounds.
\begin{bashcode}
% java Machine examples/ex3.out 1 1
20 42 
\end{bashcode}
\texttt{i=16} and \texttt{j=1}, here i is out of bounds, creating an "Array
Index Out of Bounds" error with no other output.
\begin{bashcode}
% java Machine examples/ex3.out 16 1
Array Index Out of Bounds
\end{bashcode}
\texttt{i=1} and \texttt{j=16}, here j is out of bounds but i is not therefor the output will
first be \texttt{20} followed by an Array Index Out of Bounds error.
\begin{bashcode}
% java Machine examples/ex3.out 1 16
20 Array Index Out of Bounds
\end{bashcode}

\section{Opgave 4}
\subsection{Implementation}
In the implementation I have assumed that the typs we are looking for are the
ones resulting for evaluating the access, which means that I extract the
\texttt{t} from \texttt{TypP(t)} etc.

The implementation of \texttt{tAccess} can be found below. In the case that its
an variable access, it looks up the type from the \texttt{varEnv}. In the case
of an deference access, we use the \texttt{tExpr} to evaluate the expression and returns
the pointers type. In the case of an index access we first check that the
expression is of type \texttt{TypI} and then we used the tAccess recursively to
find the access type and from that we can return the type.
\begin{fs}
let rec tExpr (e : expr) (varEnv : varEnv) (funEnv : funEnv) : typ =
    failwith "not implemented"
and tAccess (access : access) (varEnv : varEnv) (funEnv : funEnv) : typ =
    match access with
    | AccVar(s) -> 
      //Check that variable is in the varEnv and return its type
      match lookup (fst varEnv) s with
      | _ , t -> t
    | AccDeref(e) -> 
      //Check that e is of type TypP and return its type
      match tExpr e varEnv funEnv with
      | TypP(t) -> t
      | _ -> failwith "type error"
    | AccIndex(a,e) -> 
      //Check that e is of TypI and check that type of a is TypP or TypA
      match tExpr e varEnv funEnv with
      | TypI -> 
        match tAccess a varEnv funEnv with
        | TypP(t) -> t
        | TypA(t,_) -> t
        | _ -> failwith "type error"
      | _ -> failwith "type error"
\end{fs}

\pagebreak
\section{Appendix}
\subsection{mcc.fs}
The following fsharp code is an implementation of a main function that compiles
MicroC code from a file and outputs the code into an other file as well as
outputting the instruction set code to the console. The code is originally
created by me and Mikkel Larsen for use in our MicroC handins, I saw no reason
to create a new one for this exame. It is used by giving it two arguments: the
input file and the output file \texttt{./mcc.exe <input> <output>}. 
\begin{fs}
module mcc

open ParseAndComp

[<EntryPoint>]
let main(args) =
    printf "%A" (compileToFile (fromFile args.[0]) args.[1])
    0;;
\end{fs}
\subsection{Makefile}
The following is the make file I use to compile the MicroC compiler with
\texttt{mcc.fs}, it first compiles the parser and lexer and then it compiles the
fsharp code for MicroC.  The makefile can be run by GNU Make and have been used
with GNU Make 3.81 and is used by \texttt{make}.
\lstinputlisting[numbers=left, language=make, breaklines=true]{../src/Makefile}
